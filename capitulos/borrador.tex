\subsubsection{Análisis de estabilidad con Lyapunov}

% Localidad Lipschitz: Dados M, N, espacios métricos, se dice que una función  f:M\longrightarrow N es localmente lipschitz si para todo punto de M existe un entorno donde la función cumple la condición Lipschitz.

\red{¿Qué entendemos por estabilidad? definición formal.}

Centrémonos en uno de los sistema físicos más sencillos, el muelle con masa. Idealmente, la ecuación diferencial que describe su dinámica es

$$m \ddot x + kx = 0,$$

que en el espacio de estados puede ser expresada como

$$\dot x_1 = x_2, \quad \dot x_2 = - \frac{k}{m}x_1.$$

\red{Sección 3.3 del Khalil. Motivaciones e ideas fundamentales. Por qué funciona y en qué se basa.}

\red{Podemos pensar en un single integrator, con energía $1/2 ||x||^2$ ¿señal de control para que sea estable?}

\begin{definition}
Diremos que una función $f : \mathcal{D} \subset \mathds{R}^n \rightarrow \mathds{R}^n$ es localmente Lipschitz en el abierto $\mathcal{D}$ cuando, para todo punto $x_0 \in \mathcal{D}$, existe una constante $L_0 > 0$ y un entorno abierto $V_0 \subset \mathcal{D}$ del punto $x_0$ tales que

$$|f(x) - f(y)| \leq L_0 |x-y|, \quad \forall x,y \in V_0.$$
\end{definition}

Cuando una función $f(x)$ sea localmente Lipchitz en un entorno $\mathcal{D}$, entonces podremos asegurar que la derivada de $f(x)$ está acotada en $\mathcal{D}$. Teniendo esto en cuenta, se puede formalizar todo lo que hemos comentado anteriormente introduciendo el teorema de estabilidad de Lyapunov:

\begin{theorem}
    \label{th: Lyapunov}
    Sea $f(x)$ una función localmente Lipchitz en un dominio $\mathcal{D} \subset \mathds{R}^n$, que contenga al origen, y $f(0) = 0$. Dada $V(x)$ como una función continuamente diferenciable en $\mathcal{D}$, de tal modo que
    \begin{equation} \label{eq: lyap_v}
        V(0) = 0 \quad \textit{y} \quad V(x) > 0 \;\; \text{para todo} \;\; x \in \mathcal{D} \;\; con \;\; x \neq 0,
    \end{equation} 
    \begin{equation} \label{eq: lyap_vdot}
        \dot V(x) \leq 0 \;\; \text{para todo} \;\; x \in \mathcal{D}.
    \end{equation}
    Entonces el origen es un punto de equilibrio estable de $\dot x = f(x)$. Si además

    \begin{equation}
        \dot V(x) < 0 \;\; \text{para todo} \;\; x \in \mathcal{D} \;\; \text{con} \;\; x \neq 0,
    \end{equation}

    entonces el origen es asintóticamente estable. Adicionalmente, si $\mathcal{D} = \mathds{R}^n$, \eqref{eq: lyap_v} y \eqref{eq: lyap_vdot} se cumplen para todo $x \neq 0$, y
    \begin{equation}
        ||x|| \rightarrow \infty \Rightarrow V(x) \rightarrow \infty,
    \end{equation}

    entonces el origen es asintóticamente estable de forma global.
\end{theorem}

\red{Algún comentario interesante sobre este teorema...}. La demostración del \autoref{th: Lyapunov} puede consultarse en la sección 3.3 de \cite{khalil}.

%%%%%%%%%%%%%%%%%%%%%%%%%%%%%%%%%%%%%%%%%%%%%%%%%%%%%%

\subsubsection{Diseño del campo vectorial}

Una vez introducida la notación y formalizado el problema de seguimiento de caminos, el siguiente paso es diseñar un GVF que apunte a $\mathcal{P}$ basado en la minimización de del valor absoluto del error $e(p)$. Con este fin, se parte considerando la función de Lyapunov

\red{Consideramos que las curvas de nivel van a determinar la 'energía de mi sistema'.}

\begin{equation}
    V_1 (p) = \frac{1}{2} e(p)^2,
\end{equation}

cuya derivada temporal a lo largo de \eqref{eq: rover_kin} viene dada por

\begin{equation}
    \frac{d V_1(p)}{d t} = \nabla V_1(p)^T \, \dot p = e \, \nabla\varphi(p)^T 
 \, \dot p = e n^T \dot p.
\end{equation}

Inspirados por este comportamiento, los autores de \cite{gvf_classic} proponen el siguiente vector de velocidad deseada

\begin{equation}
    \dot p_d := \tau(p) - k_e e(p) n(p),
\end{equation}

donde $k_e \in \mathds{R}^+$ es una ganancia que modelará la agresividad del campo vectorial, es decir, cómo de rápido se le exige al \textit{rover} que tienda a la trayectoria. Dado este resultado, y recordando \eqref{eq: no_zero}, entonces es directo ver que

\begin{equation}
    e n^T \dot p_d = e n^T \tau(p) - e^2 k_e ||n(p)||^2 = - e^2 k_e ||n(p)||^2 \leq 0
\end{equation}

 es estrictamente decreciente siempre y cuando $e \neq 0$. Por lo tanto, teniendo en cuenta el \autoref{th: Lyapunov}, siempre y cuando $\dot p = \dot p_d$, todos los puntos de la trayectoria $\mathcal{P}$ serán puntos asintóticamente estables de \eqref{eq: rover_kin} en $\mathcal{N}_\mathcal{P}$. La componte $n$ del campo hace que el \textit{rover} se acerque a $\mathcal{P}$, mientras que $\tau$ le permite recorrer la curva de nivel $\varphi(p^*)$.