% Portada --

\begin{titlepage}
\centering
{ \bfseries \Large UNIVERSIDAD DE GRANADA}
\vspace{1cm}

{\large ESCUELA INTERNACIONAL DE POSTGRADO}
\vspace{0.8cm}

{\includegraphics[width=0.50\textwidth]{fig/UGR-Logo.png}}
\vspace{0.8cm}

{\bfseries \Large TRABAJO DE FIN DE MÁSTER}

\vfill
% {\Large Código de TFM:  ... } \vspace{5mm} \\
{\Large Diseño e integración de sistemas de Guiado Navegación y Control \\ para un enjambre de vehículos autónomos}\vspace{5mm} \\
{\Large Design and integration of Guidance, Navigation and Control systems for autonomous robot swarms}\vspace{15mm} \\
{\Large Supervisor: Héctor García de Marina}\\ 
\vfill

{\bfseries \LARGE Jesús Bautista Villar} \\
\noindent\rule{8cm}{0.4pt}\vspace{5mm}

{\large Máster Universitario en Ciencia de Datos e Ingeniería de Computadores}\vspace{2.5mm} \\
{\large Curso académico 2022-2023}\vspace{2.5mm} \\
{\large Convocatoria Julio} \\ 

\end{titlepage}

% Abstract --
%Nota: el título extendido (si procede), el resumen y el abstract deben estar en una misma página y su extensión no debe superar una página. Tamaño mínimo 11 pto.

\newpage
\thispagestyle{empty} % Página flotante para quitar numeración

Considerando que la presentación de un trabajo hecho por otra persona o la copia de
textos, fotos y gráficas sin citar su procedencia se considera plagio, el abajo firmante D. Jesús Bautista Villar con DNI 06294739F, que presenta el Trabajo Fin de Máster titulado "Diseño e integración de sistemas de Guiado Navegación y Control para un enjambre de vehículos autónomos", declara la autoría y asume la originalidad de este trabajo, donde se han utilizado
distintas fuentes que han sido todas citadas debidamente en la memoria.\\

Y para que así conste firmo el presente documento en Granada a \textbf{8 de Julio del 2023}.\\

\vspace{2cm}

{\includegraphics[trim={-7cm 0cm 0cm 0cm}, clip, width=0.3\textwidth]{fig/sign.png}}

El autor: ……………………………… .

\newpage
\thispagestyle{empty} % Página flotante para quitar numeración

{\bfseries \large Resumen:} \vspace{5mm}

En este trabajo, presentamos dos metodologías originales relacionadas con la robótica de enjambre. En primer lugar, se proponen una serie de resultados rigurosos que permiten garantizar la ausencia de colisiones durante el seguimiento de caminos con enjambres robóticos. Describimos los principios fundamentales de esta técnica y presentamos resultados rigurosos basados en el uso de ''\textit{control barrier functions}'' y ''\textit{guidance vector fields}''. Además, demostramos su efectividad en entornos reales a través de una colección de simulaciones y experimentos con una flota de \textit{rovers} autónomos.

En segundo lugar, introducimos un conjunto de herramientas resilientes para buscar fuentes de campos escalares utilizando enjambres de robots. Analizaremos rigurosamente las características principales de estas herramientas, especialmente en términos de sensibilidad y observabilidad, y discutimos los mecanismos de adaptación y resistencia frente a perturbaciones o fallas del sistema. Además, presentamos una serie de simulaciones que validan la eficacia y robustez de estas herramientas en la búsqueda de fuentes de campos escalares.

Todos los resultados obtenidos demuestran la viabilidad y eficacia de ambas metodologías en entornos reales y sientan las bases para futuras investigaciones en el campo de la robótica de enjambres.

\vspace{0.5cm}

{\bfseries Palabras clave:} Robótica de enjambre, control robusto, evasión de colisiones, búsqueda de fuentes en campos escalares.

\vspace{1cm}

{\bfseries \large Abstract: }\vspace{5mm} 

In this work, we present two innovative swarm robotics methodologies. Firstly, we propose a collision-free path-following technique for robot swarms. We describe the fundamental principles of this methodology and provide rigorous results based on the utilization of ''control barrier functions'' and ''guidance vector fields''. Through a series of simulations and experiments with an autonomous rover fleet, we demonstrate its effectiveness in real-world environments.

Secondly, we introduce a resilient approach for source-seeking using robot swarms. We conduct a thorough analysis of this methodology, considering sensitivity \& observability, and mechanisms for adaptation and resilience against disturbances or system failures. To validate its efficacy and robustness, we present a set of simulations that showcase the successful search for scalar field sources.

Overall, our results highlight the feasibility and effectiveness of both methodologies in real-world scenarios, providing a solid foundation for further research and advancements in swarm robotics.

\vspace{0.5cm}

{\bfseries  Keywords:} Swarm robotics, robust control, collision avoidance, source-seeking.

