
\usepackage[spanish]{babel} %Español 
\usepackage[utf8]{inputenc} %Para poder poner tildes
\usepackage{vmargin} %Para modificar los márgenes
\setmargins{2.5cm}{1.5cm}{16.5cm}{23.42cm}{10pt}{1cm}{0pt}{2cm}
% Margen izquierdo, superior, anchura del texto, altura del texto, altura de los encabezados, espacio entre el texto y los encabezados, altura del pie de página, espacio entre el texto y el pie de página.

\setlength{\parindent}{0cm} %Por defecto es 15pt.
% Para añadir indentación manualmente: \hangindent=0.7cm 

% Figuras, imágenes y sus captions
\usepackage{graphicx} 
\usepackage[font=small,labelfont=bf]{caption}

% Otros paquetes
\usepackage{moreverb,url}
\usepackage[colorlinks,bookmarksopen,bookmarksnumbered,citecolor=red,urlcolor=red]{hyperref}
\usepackage{cite}
\usepackage{caption}
\usepackage{subcaption}
\usepackage{floatpag} %Página del abstract no numerada
\usepackage{dsfont}
\usepackage{latexsym}
\usepackage{hyperref} %\autoref{}
\usepackage{wrapfig}  %Texto y figura "side by side"
\usepackage{tikz}
\usetikzlibrary{calc, arrows.meta}

\usepackage{amsmath}
\usepackage{amssymb}

\let\proof\relax 
\let\endproof\relax
\usepackage{amsthm}
\usepackage{mathtools}


\newcommand\norm[1]{\left\lVert#1\right\rVert}
\newcommand\normx[1]{\left\Vert#1\right\Vert}

\newtheoremstyle{theoremdd}% name of the style to be used
  {\topsep}% measure of space to leave above the theorem. E.g.: 3pt
  {\topsep}% measure of space to leave below the theorem. E.g.: 3pt
  {\itshape}% name of font to use in the body of the theorem
  {0pt}% measure of space to indent
  {\bfseries}% name of head font
  {. ---}% punctuation between head and body
  { }% space after theorem head; " " = normal interword space
  {\thmname{#1}\thmnumber{ #2}\textnormal{\thmnote{ (#3)}}}

% Estilo de los teoremos
\theoremstyle{theoremdd}
\newtheorem{theorem}{Teorema}[section]
\newtheorem{thm}[theorem]{Teorema}

\newtheorem{problema}{Problema}[section]
\newtheorem{coroll}{Colorario}[section]
\newtheorem{lemma}{Lema}[section]

\newtheorem{prop}{Proposici\'on}[section]
\newtheorem{definition}{Definici\'on}[section]

%\theoremstyle{definition}
%\newtheorem{conj}[theorem]{Conjetura}
%\newtheorem{example}[theorem]{Ejemplo}

\theoremstyle{remark}
\newtheorem{rem}{Observación\'on}[section]
\newtheorem{suposicion}{Suposición}[section]
\newtheorem{rems}[rem]{Observación\'on}

% Otras configuraciones %%%%%%%%%%%%%
\hypersetup{
    colorlinks=true,
    linkcolor=black,
    filecolor=magenta,      
    urlcolor=cyan,
    citecolor=blue,
    pdfpagemode=FullScreen,
}
\urlstyle{same}

%Tikz
\usetikzlibrary{patterns}

% Formato de numeración de las ecuaciones
\numberwithin{equation}{section}

% Macros
\renewcommand{\baselinestretch}{1} % Interlineado
\setlength{\parskip}{1.6 ex}

% Itemsize custom labels
\renewcommand{\labelenumi}{{\theenumi})}

% Math notation
\newcommand{\rank}{{\rm rank}\,}
\newcommand{\defeq}{:=} %{\stackrel{\Delta}{=}}
\newcommand{\red}[1]{{\leavevmode\color{red}#1}}
\newcommand{\blue}[1]{{\leavevmode\color{blue}#1}}
\newcommand{\green}[1]{{\leavevmode\color{green}#1}}
\newcommand{\cyan}[1]{{\leavevmode\color{cyan}#1}}

\newcommand{\dist}{{\rm dist}}
\newcommand{\dt}{\frac{{\rm d}}{{\rm d}t}}
\newcommand{\expo}[1]{{\rm exp} \left(#1\right)}					% exponential function
\newcommand{\lambdamin}[1]{\lambda_{{\rm min}}\left(#1\right)}		% minimum eigenvalue
\newcommand{\lambdamax}[1]{\lambda_{{\rm max}}\left(#1\right)}		% maximum eigenvalue
\newcommand{\lambdainf}[1]{\lambda_{{\rm inf}}\left(#1\right)}		% infimum eigenvalue
\newcommand{\lambdasup}[1]{\lambda_{{\rm sup}}\left(#1\right)}		% supremum eigenvalue
\newcommand{\matr}[1]{\begin{bmatrix} #1 \end{bmatrix}}
\newcommand{\vmatr}[1]{\begin{vmatrix} #1 \end{vmatrix}}
\newcommand{\sig}{{\rm sig}}	% sig function
\newcommand{\sgn}{{\rm sgn}}	% signum function
\newcommand\undermat[2]{%
	\makebox[0pt][l]{$\smash{\underbrace{\phantom{%
					\begin{matrix}#2\end{matrix}}}_{\text{$#1$}}}$}#2} % underbrace in matrix
\newcommand{\transpose}[1]{#1^\top}
\newcommand{\inv}[1]{#1^{-1}}	% matrix inverse
\newcommand{\diag}[1]{{\rm diag}\{ #1\}}
\newcommand{\derivative}[2]{\frac{{\rm d} #1}{{\rm d} #2}}
%\newcommand{\dgamma}[1]{\frac{{\rm d} #1}{{\rm d}\gamma}}
\newcommand{\cmmnt}[1]{}
\newcommand{\chiup}{\raisebox{2pt}{$\chi$}}
\newcommand{\vf}{\chiup}
\newcommand{\identity}{{\rm id}}
\newcommand{\set}[1]{\mathcal{#1}}

\newcommand{\normm}[1]{\big\lVert#1\big\rVert}		% for cases if \norm is not satisfactory
\newcommand{\mbr}[1][{}]{\mathbb{R}^{#1}}	% \mbr[n]

\newcommand{\proj}[1]{{#1}^{\mathrm{prj}}}
\newcommand{\trs}[1]{{#1}^{\mathrm{trs}}}
\newcommand{\hgh}[1]{{#1}^{\mathrm{hgh}}}
\newcommand{\phy}[1]{{#1}^{\mathrm{phy}}}

\newcommand{\vfpf}{\prescript{\rm pf}{}{\vf}}
\newcommand{\vfco}{\prescript{\rm co}{}{\vf}}
\newcommand{\vfcb}{\mathfrak{X}}%{\prescript{\rm a}{}{\vf}}
\newcommand{\normv}[1]{\overline{#1}}			% normalization of a vector